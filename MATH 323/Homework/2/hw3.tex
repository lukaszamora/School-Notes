\documentclass{article}

\usepackage[margin=1in]{geometry}
\usepackage{amsmath,amsthm,tikz,fancyhdr,bm,enumitem,amssymb}


\theoremstyle{definition}

\newtheorem{innercustomgeneric}{\customgenericname}
\providecommand{\customgenericname}{}
\newcommand{\newcustomtheorem}[2]{%
  \newenvironment{#1}[1]
  {%
   \renewcommand\customgenericname{#2}%
   \renewcommand\theinnercustomgeneric{##1}%
   \innercustomgeneric
  }
  {\endinnercustomgeneric}
}

\newcustomtheorem{prob}{Problem}
\newcustomtheorem{customlemma}{Lemma}

\pagestyle{fancy}
\fancyhf{}
\rhead{Lukas Zamora}
\chead{Homework 3}
\lhead{MATH 323}
\cfoot{\thepage}

\title{MATH 323 - Homework 3}
\author{Lukas Zamora}
\date{September 6, 2018}

\setlength\parindent{0pt}

\renewcommand{\vec}[1]{\boldsymbol{\mathbf{#1}}}
\DeclareMathOperator{\proj}{proj}


\begin{document}

    \maketitle
    
    \begin{prob}{1.3.14} $  $ \\
		\begin{proof}
			Suppose that $ || \proj_{\vec{x}} \vec{y} || = \vec{x} \cdot \vec{y} $. Since $ \cos\theta = \frac{\vec{x} \cdot \vec{y}}{||\vec{x}|| \, ||\vec{y}||} $, we have  
			\begin{align*}
				|| \proj_{\vec{x}} \vec{y} || = ||\vec{x}|| \, ||\vec{y}|| \cos\theta &\Leftrightarrow \\
				|| (\vec{x} \cdot \vec{y}) \vec{x} || = ||\vec{x}|| \, ||\vec{y}|| \cos\theta &\Leftrightarrow \\
				||\vec{x}||^2 \cdot \vec{y} = ||\vec{x}|| \, ||\vec{y}|| \cos\theta &\Leftrightarrow \\
				||\vec{x}|| = \cos\theta
			\end{align*}
			which is a contradiction. Thus $ \vec{x} $ is not a unit vector.
		\end{proof}    	
    \end{prob}

	\begin{prob}{1.4.3} $  $ \\ \\
		We have the fact that $ A = S + V $, where $ S = \frac{1}{2}(A+A^T) $ and $ V = \frac{1}{2}(A-A^T) $.
		\begin{enumerate}[label=(\alph*)]
			\item $ A = \begin{bmatrix} 3 & -1 & 4 \\ 0 & 2 & 5 \\ 1 & -3 & 2 \end{bmatrix} $
				\begin{align*}
					S = \frac{1}{2}(A+A^T) &= \frac{1}{2} \left( \begin{bmatrix} 3 & -1 & 4 \\ 0 & 2 & 5 \\ 1 & -3 & 2 \end{bmatrix} + \begin{bmatrix} 3 & 0 & 1 \\ -1 & 2 & -3 \\ 4 & 5 & 2 \end{bmatrix} \right) \\
					&= \frac{1}{2} \begin{bmatrix} 6 & -1 & 5 \\ -1 & 4 & 2 \\ 5 & 2 & 4 \end{bmatrix}
					= \begin{bmatrix} 3 & -1/2 & 5/2 \\ -1/2 & 2 & 1 \\ 5/2 & 1 & 2 \end{bmatrix} \\ \\
					V = \frac{1}{2}(A-A^T) &= \frac{1}{2} \left( \begin{bmatrix} 3 & -1 & 4 \\ 0 & 2 & 5 \\ 1 & -3 & 2 \end{bmatrix} - \begin{bmatrix} 3 & 0 & 1 \\ -1 & 2 & -3 \\ 4 & 5 & 2 \end{bmatrix} \right) \\
					&= \frac{1}{2} \begin{bmatrix} 0 & -1 & 3 \\ 1 & 3 & 6 \\ -3 & -8 & 0 \end{bmatrix} = \begin{bmatrix} 0 & -1/2 & 3/2 \\ 1/2 & 3/2 & 3 \\ -3/2 & -4 & 0 \end{bmatrix}
				\end{align*}
				$$ \therefore \boxed{ A = \begin{bmatrix} 3 & -1/2 & 5/2 \\ -1/2 & 2 & 1 \\ 5/2 & 1 & 2 \end{bmatrix} + \begin{bmatrix} 0 & -1/2 & 3/2 \\ 1/2 & 3/2 & 3 \\ -3/2 & -4 & 0 \end{bmatrix} } $$
				
			\item $ A = \begin{bmatrix} 1 & 0 & -4 \\ 3 & 3 & -1 \\ 4 & -1 & 0 \end{bmatrix} $
				\begin{align*}
					S = \frac{1}{2}(A+A^T) &= \frac{1}{2} \left( \begin{bmatrix} 1 & 0 & -4 \\ 3 & 3 & -1 \\ 4 & -1 & 0 \end{bmatrix} + \begin{bmatrix} 1 & 3 & 4 \\ 0 & 3 & -1 \\ -4 & -1 & 0 \end{bmatrix} \right) \\
					&= \frac{1}{2} \begin{bmatrix} 2 & 3 & 0 \\ 3 & 6 & -2 \\ 0 & -2 & 0 \end{bmatrix}
					= \begin{bmatrix} 1 & 3/2 & 0 \\ 3/2 & 3 & -1 \\ 0 & -1 & 0 \end{bmatrix} \\ \\
					V = \frac{1}{2}(A-A^T) &= \left( \begin{bmatrix} 1 & 0 & -4 \\ 3 & 3 & -1 \\ 4 & -1 & 0 \end{bmatrix} - \begin{bmatrix} 1 & 3 & 4 \\ 0 & 3 & -1 \\ -4 & -1 & 0 \end{bmatrix} \right) \\
					&= \frac{1}{2} \begin{bmatrix} 0 & -3 & -8 \\ 3 & 0 & 0 \\ 8 & 0 & 0 \end{bmatrix} = \begin{bmatrix} 0 & -3/2 & -4 \\ 3/2 & 0 & 0 \\ 4 & 0 & 0 \end{bmatrix}
				\end{align*}
				$$ \therefore \boxed{ A = \begin{bmatrix} 1 & 3/2 & 0 \\ 3/2 & 3 & -1 \\ 0 & -1 & 0 \end{bmatrix} + \begin{bmatrix} 0 & -3/2 & -4 \\ 3/2 & 0 & 0 \\ 4 & 0 & 0 \end{bmatrix} } $$
				
			\item $ A = \begin{bmatrix} 2 & 3 & 4 & -1 \\ -3 & 5 & -1 & 2 \\ -4 & 1 & -2 & 0 \\ 1 & -2 & 0 & 5 \end{bmatrix} $
				\begin{align*}
					S = \frac{1}{2}(A+A^T) &= \frac{1}{2} \left( \begin{bmatrix} 2 & 3 & 4 & -1 \\ -3 & 5 & -1 & 2 \\ -4 & 1 & -2 & 0 \\ 1 & -2 & 0 & 5 \end{bmatrix} + \begin{bmatrix} 2 & -3 & -4 & 1 \\ 3 & 5 & 1 & -2 \\ 4 & -1 & -2 & 0 \\ -1 & 2 & 0 & 5 \end{bmatrix} \right) \\
					&= \frac{1}{2} \begin{bmatrix} 4 & 0 & 0 & 0 \\ 0 & 10 & 0 & 0 \\ 0 & 0 & -4 & 0 \\ 0 & 0 & 0 & 10 \end{bmatrix} = \begin{bmatrix} 2 & 0 & 0 & 0 \\ 0 & 5 & 0 & 0 \\ 0 & 0 & -2 & 0 \\ 0 & 0 & 0 & 5 \end{bmatrix} \\ \\
					V = \frac{1}{2}(A-A^T) &= \left( \begin{bmatrix} 2 & 3 & 4 & -1 \\ -3 & 5 & -1 & 2 \\ -4 & 1 & -2 & 0 \\ 1 & -2 & 0 & 5 \end{bmatrix} - \begin{bmatrix} 2 & -3 & -4 & 1 \\ 3 & 5 & 1 & -2 \\ 4 & -1 & -2 & 0 \\ -1 & 2 & 0 & 5 \end{bmatrix} \right) \\
					&= \frac{1}{2} \begin{bmatrix} 0 & 6 & 2 & -2 \\ -6 & 0 & -2 & 4 \\ -8 & 2 & 0 & 0 \\ 2 & -4 & 0 & 0 \end{bmatrix} = \begin{bmatrix} 0 & 3 & 4 & -1 \\ -3 & 0 & -1 & 2 \\ -4 & 1 & 0 & 0 \\ 1 & -2 & 0 & 0 \end{bmatrix}
				\end{align*}
				$$ \therefore \boxed{ A = \begin{bmatrix} 2 & 0 & 0 & 0 \\ 0 & 5 & 0 & 0 \\ 0 & 0 & -2 & 0 \\ 0 & 0 & 0 & 5 \end{bmatrix} + \begin{bmatrix} 0 & 3 & 4 & -1 \\ -3 & 0 & -1 & 2 \\ -4 & 1 & 0 & 0 \\ 1 & -2 & 0 & 0 \end{bmatrix} } $$	
				
			\item $ A = \begin{bmatrix} -3 & 3 & 5 & -4 \\ 11 & 4 & 5 & -1 \\ -9 & 1 & 5 & -14 \\ 2 & -11 & -2 & -5 \end{bmatrix} $
				\begin{align*}
					S = \frac{1}{2}(A+A^T) &= \frac{1}{2} \left( \begin{bmatrix} -3 & 3 & 5 & -4 \\ 11 & 4 & 5 & -1 \\ -9 & 1 & 5 & -14 \\ 2 & -11 & -2 & -5 \end{bmatrix} + \begin{bmatrix} -3 & 11 & -9 & 2 \\ 3 & 4 & 1 & -11 \\ 5 & 5 & 5 & -2 \\ -4 & -1 & -14 & -5 \end{bmatrix} \right) \\
					&= \frac{1}{2} \begin{bmatrix} -6 & 14 & -4 & -2 \\ 14 & 8 & 6 & -12 \\ -4 & 6 & 10 & -16 \\ -2 & -12 & -16 & -10 \end{bmatrix} = \begin{bmatrix} -3 & 7 & -2 & -1 \\ 7 & 4 & 3 & -6 \\ -2 & 3 & 5 & -8 \\ -1 & -6 & -8 & -5 \end{bmatrix} \\ \\
					V = \frac{1}{2}(A-A^T) &= \frac{1}{2} \left( \begin{bmatrix} -3 & 3 & 5 & -4 \\ 11 & 4 & 5 & -1 \\ -9 & 1 & 5 & -14 \\ 2 & -11 & -2 & -5 \end{bmatrix} - \begin{bmatrix} -3 & 11 & -9 & 2 \\ 3 & 4 & 1 & -11 \\ 5 & 5 & 5 & -2 \\ -4 & -1 & -14 & -5 \end{bmatrix} \right) \\
					&= \frac{1}{2} \begin{bmatrix} 0 & -8 & 14 & -6 \\ 8 & 0 & 4 & 10 \\ -14 & -4 & 0 & -12 \\ 6 & -10 & 12 & 0 \end{bmatrix} = \begin{bmatrix} 0 & -4 & 7 & -3 \\ 4 & 0 & 2 & 5 \\ -7 & -2 & 0 & -6 \\ 3 & -5 & 6 & 0 \end{bmatrix}
				\end{align*}
				$$ \therefore \boxed{ A = \begin{bmatrix} -3 & 7 & -2 & -1 \\ 7 & 4 & 3 & -6 \\ -2 & 3 & 5 & -8 \\ -1 & -6 & -8 & -5 \end{bmatrix} + \begin{bmatrix} 0 & -4 & 7 & -3 \\ 4 & 0 & 2 & 5 \\ -7 & -2 & 0 & -6 \\ 3 & -5 & 6 & 0 \end{bmatrix} } $$
		\end{enumerate}
	\end{prob}

	\begin{prob}{1.5.2b}
		\begin{align*}
			&GH = \begin{bmatrix} 5 & 1 & 0 \\ 0 & -2 & -1 \\ 1 & 0 & 3 \end{bmatrix} \cdot \begin{bmatrix} 6 & 3 & 1 \\ 1 & -15 & -5 \\ -2 & -1 & 10 \end{bmatrix} = \begin{bmatrix} 31 & 0 & 0 \\ 0 & 31 & 0 \\ 0 & 0 & 31 \end{bmatrix} \\
			&HG = \begin{bmatrix} 6 & 3 & 1 \\ 1 & -15 & -5 \\ -2 & -1 & 10 \end{bmatrix} \cdot \begin{bmatrix} 5 & 1 & 0 \\ 0 & -2 & -1 \\ 1 & 0 & 3 \end{bmatrix} = \begin{bmatrix} 31 & 0 & 0 \\ 0 & 31 & 0 \\ 0 & 0 & 31 \end{bmatrix}
		\end{align*}
		Since $ GH = HG $, the matrices $ G $ and $ H $ commute. \vspace{1mm}
	\end{prob}
	
	
	\begin{prob}{1.5.2e} $  $ \\ \\
		$ FQ $ will have size $ 4 \times 4 $ while $ QF $ will have size $ 2 \times 2 $. Thus the matrices $ F $ and $ Q $ do not commute. \vspace{1mm}
	\end{prob}

	\begin{prob}{1.5.23} $  $ \vspace{1mm}
		\begin{enumerate}[label=(\alph*)]
			\item
				\begin{proof}
					Let $ A \in \mathbf{R}^{m \times n}, B \in \mathbf{R}^{n \times p} $. So $ AB \in \mathbf{R}^{m \times p} $. However, if $ m=n=p $, we can conclude that $ BA \in \mathbf{R}^{p \times m} $. Thus $ AB $ is commutative.
				\end{proof}
			\item 
				\begin{proof}
					Let $ A \in \mathbf{R}^{m \times m}, B \in \mathbf{R}^{n \times n} $. From (a), we know that $ AB $ is commutative, i.e. $ AB = BA $. We then have
						\begin{align*}
							(A+B)^2 &= A^2 + AB + BA + B^2 \\
									&= A^2 + 2AB + B^2
						\end{align*}
				\end{proof}
		\end{enumerate}
	\end{prob}
	
	
	\begin{prob}{Ch. 1 Review 19} $  $ \\
		\begin{proof}
			By Mathematical Induction. \\
			
			\underline{Base Case}: $ (k=2) $. Suppose $ A $ and $ B $ are upper triangular $ n \times n $ matrices, and let $ C = AB $. Then $ a_{ij} = b_{ij} = 0 $ for $ i>j $. Hence for $ i>j $,
			\[
				c_{ij} = \sum\limits_{m=1}^{n} a_{im} b_{mj} = \sum\limits_{m=1}^{i-1} 0 \cdot b_{mj} + a_{ii} b_{ij} + \sum\limits_{m=i+1}^{n} a_{im} \cdot 0 = a_{ii}(0) = 0
			\]
			Thus $ C $ is upper triangular. \\
			
			\underline{Inductive Step}: Let $ A_1, A_2, \dots, A_{k+1} $ be upper triangular matrices. Then the product $ C = A_1 A_2 \dots A_{k} $ is upper triangular by the inductive hypothesis, and so the product $ A_1 A_2 \dots A_{k+1} = CA_{k+1} $ is upper triangular by the base step. \\
		\end{proof}
	\end{prob}




















\end{document}