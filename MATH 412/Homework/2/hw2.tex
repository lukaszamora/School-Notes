\documentclass{article}

\usepackage[margin=1in]{geometry}
\usepackage{amsmath,amsthm,tikz,fancyhdr,bm,enumitem,amssymb}


\theoremstyle{definition}

\newtheorem{innercustomgeneric}{\customgenericname}
\providecommand{\customgenericname}{}
\newcommand{\newcustomtheorem}[2]{%
  \newenvironment{#1}[1]
  {%
   \renewcommand\customgenericname{#2}%
   \renewcommand\theinnercustomgeneric{##1}%
   \innercustomgeneric
  }
  {\endinnercustomgeneric}
}

\newcustomtheorem{prob}{Problem}
\newcustomtheorem{customlemma}{Lemma}

\pagestyle{fancy}
\fancyhf{}
\rhead{Lukas Zamora}
\chead{Homework 2}
\lhead{MATH 412}
\cfoot{\thepage}

\title{MATH 412 - Homework 2}
\author{Lukas Zamora}
\date{September 14, 2018}

\setlength\parindent{0pt}



\begin{document}

    \maketitle
    
    \begin{prob}{2.2.2} $  $ \\
    	\begin{enumerate}[label=\alph*.)]
    		\item $ L(u) = \dfrac{\partial}{\partial x} \left( K_0(x) \dfrac{\partial u}{\partial x} \right) = \dfrac{dK_0}{dx} \dfrac{\partial u}{\partial x} + K_0(x) \dfrac{\partial^2 u}{\partial x^2} $
    		\begin{align*}
    			L(c_1 u_1 + c_2 u_2) &= \frac{dK_0}{dx} \frac{\partial}{\partial x} (c_1 u_1 + c_2 u_2) + K_0(x) \frac{\partial^2}{\partial x^2} (c_1 u_1 + c_2 u_2) \\
    			&= \frac{dK_0}{dx} \left( c_1 \frac{\partial u_1}{\partial x} + c_2 \frac{\partial u_2}{\partial x} \right) + K_0(x) \left( c_1 \frac{\partial^2 u_1}{\partial x^2} + c_2 \frac{\partial^2 u_2}{\partial x^2} \right) \\
    			&= c_1 \frac{dK_0}{dx} \frac{\partial u_1}{\partial x} + c_2 \frac{dK_0}{dx} \frac{\partial u_2}{\partial x} + c_1 K_0(x) \frac{\partial^2 u_1}{\partial x^2} + c_2 K_0(x) \frac{\partial^2 u_2}{\partial x^2} \\
    			&= c_1 \frac{dK_0}{dx} \frac{\partial u_1}{\partial x} + c_1 K_0(x) \frac{\partial^2 u_1}{\partial x^2} + c_2 \frac{dK_0}{dx} \frac{\partial u_2}{\partial x} + c_2 K_0(x) \frac{\partial^2 u_2}{\partial x^2} \\
    			&= c_1 L(u_1) + c_2 L(u_2)
    		\end{align*}
    		
    		\item $ L(u) = \dfrac{\partial}{\partial x} \left( K_0(x,u) \dfrac{\partial u}{\partial x} \right) $
    		\begin{align*}
    			L(c_1 u_1 + c_2 u_2) &= \frac{\partial}{\partial x} \left( K_0(x, c_1 u_1 + c_2 u_2) \frac{\partial}{\partial x}(c_1 u_1 + c_2 u_2) \right) \\
    			&= \frac{\partial}{\partial x} \left( K_0(x, c_1 u_1 + c_2 u_2) \left( c_1 \frac{\partial u_1}{\partial x} + c_2 \frac{\partial u_2}{\partial x} \right) \right) \\
    			&= \frac{\partial}{\partial x} \left( c_1 K_0(x, c_1 u_1 + c_2 u_2) \frac{\partial u_1}{\partial x} + c_2 K_0(x, c_1 u_1 + c_2 u_2) \frac{\partial u_2}{\partial x} \right) \\
    			&\neq c_1 \frac{\partial}{\partial x} \left( K_0(x,u_1) \frac{\partial u_1}{\partial x} \right) + c_2 \frac{\partial}{\partial x} \left( K_0(x,u_2) \frac{\partial u_2}{\partial x} \right)
    		\end{align*}
    	\end{enumerate}
    \end{prob}

	\begin{prob}{2.3.2c} $  $ \vspace{2mm} \\
		$ \dfrac{d^2 \phi}{dx^2} + \lambda \phi = 0, \quad \dfrac{d\phi}{dx}(0) = \dfrac{d\phi}{dx}(L) = 0 $ \\
		
		$ r^2 + \lambda = 0 \; \Rightarrow \; r = \pm \sqrt{-\lambda} $ \\
		
		If $ \lambda > 0 $
		\begin{align*}
			& \phi(x) = c_1\cos(\sqrt{\lambda} x) + c_2\sin(\sqrt{\lambda} x) \\
			& \phi'(x) = -c_1\sqrt{\lambda} \sin(\sqrt{\lambda}) + c_2\sqrt{\lambda} \cos(\sqrt{\lambda} x) \\
			& \phi'(0) = 0 = c_2\sqrt{\lambda} \; \Rightarrow \; c_2 = 0 \\
			& \phi'(L) = 0 = -c_1\sin(\sqrt{\lambda} L) \\
			& \hspace{0.84cm}  \Rightarrow \lambda = \left( \frac{n\pi}{L} \right)^2 \, , \, n = \pm 1, \pm 2, \dots
		\end{align*}
		If $ \lambda = 0 $
		\begin{align*}
			& \phi(x) = c_1x + c_2 \\
			& \phi'(x) = c_1 \\
			& \phi'(0) = 0 = c_2 \\
			& \phi'(L) = 0 = c_1 \\
			& \hspace{0.84cm} \Rightarrow \lambda = 0
		\end{align*}
		If $ \lambda < 0 $
		\begin{align*}
			& \phi(x) = c_1\cosh(\sqrt{-\lambda} x) + c_2\sinh(\sqrt{-\lambda} x) \\
			& \phi'(x) = -c_1\sqrt{-\lambda} \sinh(\sqrt{-\lambda} x) + c_2 \sqrt{-\lambda} \cosh(\sqrt{-\lambda} x) \\
			& \phi'(0) = 0 = c_2 \\
			& \phi'(L) = 0 = -c_1\sqrt{-\lambda}\sinh(\sqrt{-\lambda} x) \; \Rightarrow c_1 = 0
		\end{align*}
		There are no eigenvalues for $ \lambda < 0 $.
	\end{prob}
    	
    \begin{prob}{2.3.2e} $ $  \vspace{2mm} \\
    	$ \dfrac{d^2 \phi}{dx^2} + \lambda \phi = 0, \quad \dfrac{d\phi}{dx}(0) = \phi(L) = 0 $ \\
    	
    	$ r^2 + \lambda = 0 \; \Rightarrow \; r = \pm \sqrt{-\lambda} $ \\
    	
    	If $ \lambda > 0 $
    	\begin{align*}
    		& \phi(x) = c_1\cos(\sqrt{\lambda} x) + c_2\sin(\sqrt{\lambda} x) \\
    		& \phi'(x) = -c_1\sqrt{\lambda} \sin(\sqrt{\lambda}) + c_2\sqrt{\lambda} \cos(\sqrt{\lambda} x) \\
    		& \phi'(0) = 0 = c_2 \\
    		& \phi(L) = 0 = c_1\cos(\sqrt{\lambda} L) + c_2\sin(\sqrt{\lambda} L) \\
    		& \hspace{0.84cm}  \Rightarrow \sqrt{\lambda}L = (2n-1)\frac{\pi}{2} \; \Rightarrow \; \lambda = \left( \frac{(2n-1)\pi}{2L} \right)^2 \, , \, n = \pm 1, \pm 2, \dots
    	\end{align*}
    	If $ \lambda = 0 $
    	\begin{align*}
    		& \phi(x) = c_1x + c_2 \\
    		& \phi'(x) = c_1 \\
    		& \phi'(0) = 0 = c_1 \\
    		& \phi(L) = 0 = c_1L + c_2 \\
    		& \hspace{0.84cm} \Rightarrow c_2 = 0
    	\end{align*}
    	There are no eigenvalues for $ \lambda = 0 $. \\
    	
    	If $ \lambda < 0 $
    	\begin{align*}
    		& \phi(x) = c_1\cosh(\sqrt{-\lambda} x) + c_2\sinh(\sqrt{-\lambda} x) \\
    		& \phi'(x) = -c_1\sqrt{-\lambda} \sinh(\sqrt{-\lambda} x) + c_2 \sqrt{-\lambda} \cosh(\sqrt{-\lambda} x) \\
    		& \phi'(0) = 0 = c_2 \\
    		& \phi(L) = 0 = c_1\cosh(\sqrt{-\lambda} L) + c_2\sinh(\sqrt{-\lambda} L) \; \Rightarrow \; c_1 = 0
    	\end{align*}
    	There are no eigenvalues for $ \lambda < 0 $.
    \end{prob}
    
    \begin{prob}{2.3.3a} $  $ \vspace{1mm} \\
    	$ u(x,0) = f(x) = 6\sin\left( \dfrac{9\pi x}{L} \right), \quad u(0,t) = u(L,t) = 0 $ \\
    	
    	By the principle of superposition,
    	\[
    		u(x,0) = \sum\limits_{n=1}^{\infty} B_n e^{-(n\pi/L)^2kt} \sin\left( \frac{n\pi x}{L} \right) = 6\sin\left( \frac{9\pi x}{L} \right)
    	\]
    	which implies that $ B_9 = 6 $. Thus,
    	\[
    		\boxed{ u(x,t) = 6\sin\left( \frac{9\pi x}{L} \right) e^{-(9\pi/L)^2kt}  }
    	\]
    \end{prob}
    
    	
    \begin{prob}{2.3.3b} $ $ \vspace{1mm} \\
    	Starting with $f(x),$
    	\begin{align*}
    		u(x,0) = f(x) &= 3\sin\left( \dfrac{\pi x}{L} \right) - \sin\left( \dfrac{3\pi x}{L} \right) = \sum\limits_{n=1}^{\infty} B_n \sin\left( \frac{n\pi x}{L} \right) \\
    		& \Rightarrow B_1 = 3, B_3 = -1
    	\end{align*}
    	By the principle of superposition, our solution becomes
    	\[
    		\boxed{u(x,t) = 3\sin\left( \frac{\pi x}{L} \right) e^{-(n\pi/L)^2 kt} - \sin\left( \frac{3\pi x}{L} \right) e^{-(n\pi/L)^2kt} }
    	\]
    \end{prob}
    
    \begin{prob}{2.3.5} $ $ \vspace{1mm} \\
    	Using the fact that $ \sin(a) \sin(b) = \frac{1}{2}(\cos(a-b) - \cos(a+b)) $,
    	\[
    		\sin\left( \frac{n\pi x}{L} \right) \sin\left( \frac{m\pi x}{L} \right) = \frac{1}{2} \left( \cos\left( \frac{(n-m)\pi x}{L} \right) - \cos\left( \frac{(n+m)\pi x}{L} \right)  \right)
    	\]
    	We then have the following 2 cases: \\
    	
    	\underline{Case 1}: $n=m\neq 0$
    	\begin{align*}
    		\int_0^L \sin\left( \frac{n\pi x}{L} \right) \sin\left( \frac{m\pi x}{L} \right) \, dx & = \int_0^L \frac{1}{2} \left( 1 - \cos\left( \frac{2n\pi x}{L} \right) \right) \, dx \\
    		& = \frac{1}{2} \left( x  \, \Big|_{0}^{L} - \frac{L}{2n\pi} \sin\left( \frac{2n\pi x}{L} \right)_{0}^{L} \right) \\
    		& = \frac{L}{2}
    	\end{align*}
    	
    	\underline{Case 2}: $ n \neq m $
    	\begin{align*}
    		\int_0^L \sin\left( \frac{n\pi x}{L} \right) \sin\left( \frac{m\pi x}{L} \right) \, dx & = \frac{1}{2} \left( \int_0^L \cos\left( \frac{(n-m)\pi x}{L} \right) \, dx - \int_0^L \cos\left( \frac{(n+m)\pi x}{L} \right) \, dx \right) \\
    		& = \frac{1}{2} \left( \frac{L}{(n-m)\pi} \sin\left( \frac{(n-m)\pi x}{L} \right)_{0}^{L} - \frac{L}{(n+m)\pi} \sin\left( \frac{(n+m)\pi x}{L} \right)_{0}^{L} \right) \\
    		& = \frac{L}{2(n-m)\pi} (\sin((n-m)\pi) - \sin(0)) - \frac{L}{2(n+m)\pi} (\sin((n+m)\pi) - \sin(0)) \\
    		&= 0
    	\end{align*}
    	\[
    		\therefore \boxed{ \displaystyle \int_0^L \sin\left( \frac{n\pi x}{L} \right) \sin\left( \frac{m\pi x}{L} \right) \, dx = \begin{cases} L/2 \quad n=m \\ 0 \qquad \, n \neq m \end{cases} }
    	\]
    \end{prob}
    	
    	
    \begin{prob}{2.4.3} $  $ \vspace{2mm} \\
    	$ \dfrac{d^2 \phi}{dx^2} + \lambda \phi = 0, \quad \phi(0) = \phi(2\pi)  , \quad \dfrac{d\phi}{dx}(0) = \dfrac{d\phi}{dx}(2\pi) $ \\
    	
    	$ r^2 + \lambda = 0 \; \Rightarrow \; r = \pm \sqrt{-\lambda} $ \\
    	
    	If $ \lambda > 0 $
    	\begin{align*}
    		& \phi(x) = c_1\cos(\sqrt{\lambda} x) + c_2\sin(\sqrt{\lambda} x) \\
    		& \phi(0) = c_1 = c_1\cos(\sqrt{\lambda} 2\pi) + c_2\sin(\sqrt{\lambda} 2\pi) \\
    		& \phi'(x) = -c_1\sqrt{\lambda} \sin(\sqrt{\lambda} x) + c_2\sqrt{\lambda} \cos(\sqrt{\lambda} x) \\
    		& \phi'(0) = c_2 = -c_1\sqrt{\lambda} \sin(\sqrt{\lambda} 2\pi) + c_2\sqrt{\lambda} \cos(\sqrt{\lambda} 2\pi) \\
    		& \hspace{0.84cm}  \Rightarrow \sqrt{\lambda} 2\pi = 1 \\
    		& \hspace{0.84cm} \Rightarrow \lambda = n^2 , \, n = \pm 1, \pm 2, \dots
    	\end{align*}
    	If $ \lambda = 0 $
    	\begin{align*}
    		& \phi(x) = c_1x + c_2 \\
			& \phi(0) = 0 = c_1 \Rightarrow c_2 = 0
    	\end{align*}
    	There are no eigenvalues for $ \lambda = 0 $
    	If $ \lambda < 0 $
    	\begin{align*}
    		& \phi(x) = c_1\cosh(\sqrt{-\lambda} x) + c_2\sinh(\sqrt{-\lambda} x) \\
    		& \phi(0) = 0 \; \Rightarrow \; c_1 = c_2 = 0
    	\end{align*}
    	There are no eigenvalues for $ \lambda < 0 $.
    \end{prob}

	\begin{prob}{2.4.4} $  $ \vspace{2mm} \\
		$ \dfrac{d^2 \phi}{dx^2} + \lambda \phi = 0, \quad \dfrac{d\phi}{dx}(0) = \dfrac{d\phi}{dx}(L) = 0 $ \\
		
		If $ \lambda < 0 $
		\begin{align*}
			& \phi(x) = c_1\cosh(\sqrt{-\lambda} x) + c_2 \sinh(\sqrt{-\lambda} x) \\
			& \phi'(x) = -c_1\sqrt{-\lambda} \sinh(\sqrt{-\lambda} x) + c_2 \sqrt{-\lambda} \cosh(\sqrt{-\lambda} x) \\
			& \phi'(0) \; \Rightarrow \; c_2 = -c_1 \\
			& \Rightarrow \sin(\sqrt{-\lambda} L) > 0 \text{ so } c_1 = 0 \Rightarrow \phi(x) = 0
		\end{align*}
		So there are no negative eigenvalues.
	\end{prob}

	\begin{prob}{2.4.6} $  $
		\begin{enumerate}[label=\alph*.)]
			\item For equilibrium, we know that $ \partial u / \partial t = 0 $. So
			\[
				\frac{d^2 u}{dx^2} = 0 \Rightarrow u(x) = c_1x + c_2
			\]
			From $ u(-L) = u(L) $, we have
			\[
				-c_1L + c_2 = c_1L + c_2	
			\]
			Thus $ c_1 = 0 $. The second condition $ u'(-L) = u'(L) $ yields $ c_1 = c_1 $. Therefore the equilibrium solution is
			\[
				u(x) = c_2
			\]
			If a system is in equilibrium, the total energy is constant, and the initial temperature is equal to the final temperature, i.e, $ u(x,0) = f(x) $ and $ u(x) = c_2 $. So we have
			\begin{align*}
				\int_{-L}^{L} f(x) \, dx & = \int_{-L}^{L} c_2 \, dx \\
				&= 2Lc_2 \\ 
				& \Rightarrow c_2 = \frac{1}{2L} \int_{-L}^{L} f(x) \, dx 
			\end{align*}
			Therefore the equilibrium temperature distribution is
			\[
				u(x) = \frac{1}{2L} \int_{-L}^{L} f(x) \, dx 
			\]
			
			\item By equation (2.4.38), we know that the solution to the time dependent problem is
			\[
				u(x,t) = a_0 + \sum\limits_{n=1}^{\infty} a_n \cos\left( \frac{n\pi x}{L} \right) e^{-(n\pi/L)^2 kt} + \sum\limits_{n=1}^{\infty} b_n \sin\left( \frac{n\pi x}{L} \right) e^{-(n\pi/L)^2kt}
			\]
			and by equation (2.4.43),
			\[
				a_0 = \frac{1}{2L} \int_{-L}^{L} f(x) \, dx
			\]
			Since $ \lim\limits_{t \to \infty} u(x,t) = a_0 $, this implies that 
			\[
				u(x) = \frac{1}{2L} \int_{-L}^{L} f(x) \, dx
			\]
		\end{enumerate}
	\end{prob}
    	

    
    	
    	
    	
    	
    	
    	
    	
\end{document}
