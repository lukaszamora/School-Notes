\documentclass{article}

\usepackage[margin=1in]{geometry}
\usepackage{amsmath,amsthm,tikz,fancyhdr,bm,enumitem,amssymb,empheq}
\usepackage{esint}

\theoremstyle{definition}

\newcommand*\widefbox[1]{\fbox{\hspace{2em}#1\hspace{2em}}}

\newtheorem{innercustomgeneric}{\customgenericname}
\providecommand{\customgenericname}{}
\newcommand{\newcustomtheorem}[2]{%
  \newenvironment{#1}[1]
  {%
   \renewcommand\customgenericname{#2}%
   \renewcommand\theinnercustomgeneric{##1}%
   \innercustomgeneric
  }
  {\endinnercustomgeneric}
}

\newcustomtheorem{prob}{Problem}
\newcustomtheorem{customlemma}{Lemma}

\pagestyle{fancy}
\fancyhf{}
\rhead{Lukas Zamora}
\chead{Homework 7}
\lhead{MATH 412}
\cfoot{\thepage}

\title{MATH 412 -- Homework 7}
\author{Lukas Zamora}
\date{October 26, 2018}

\setlength\parindent{0pt}


\begin{document}

    \maketitle
    
    \begin{prob}{12.4.1} $  $ \vspace{3mm} \\
    	$ \dfrac{\partial^2 u}{\partial t^2} = c^2 \dfrac{\partial^2 u}{\partial x^2}, \qquad u(x,0) = \dfrac{\partial u}{\partial t}(x,0) = 0, \quad u(0,t) = h(t) $ \\
    	
    	The solution is of the form
    	\[ u(x,t) = F(x-ct) + G(x+ct) \]
    	For $ x>0 $, the initial condition yields
    	\[ F(x) = G(x) = 0 \quad x>0 \]
    	And for $ t>0 $, the boundary condition yields
    	\[ h(t) = F(-ct) + G(ct) \quad t>0 \]
    	So if $ x>ct $, $ F,G $ are both positive thus $ F=G=0 \Rightarrow u(x,t) = 0 $. If $ x<ct $, then $ F $ is negative, so we have
    	\begin{align*}
    		u(x,t) &= F(x-ct) + G(x+ct) \\
    			   &= h\left( t - \frac{x}{c} \right) - G(t-x) + G(ct+x) 
    	\end{align*}
    	Since $ x<ct $, both $ ct-x>0 $ and $ ct+x>0 $. Thus
    	\[
    		\boxed{ u(x,t) = \begin{cases} 0 \hspace{1.75cm} x>ct \\ h\left( t - \dfrac{x}{c} \right) \quad x<ct \end{cases} }
    	\]
    \end{prob}
    
    \begin{prob}{12.4.2} $  $ \vspace{3mm} \\
    	$ \dfrac{\partial^2 u}{\partial t^2} = c^2 \dfrac{\partial^2 u}{\partial x^2}, \qquad u(x,0) = \cos(x), \quad \dfrac{\partial u}{\partial t}(x,0) = 0, \quad u(0,t) = e^{-t} $ \\
    	
    	The solution is of the form
    	\[ u(x,t) = F(x-ct) + G(x+ct) \]
    	For $ x<0 $, the initial condition yields
    	\[ \cos(x) = F(x) + G(x) \]
    	implying that $ F(x) = G(x) = \frac{1}{2}\cos(x) $. For $ x<-ct $, we have
    	\begin{align*}
    		u(x,t) &= \frac{1}{2}\cos(x-ct) + \frac{1}{2}\cos(x+ct) \\
    			   &= \frac{1}{2} (2\cos(x-ct)\cos(x+ct)) \\
    			   &= \cos(x)\cos(ct) 
    	\end{align*}
    	If $ x+ct > 0 $, then both $ F,G $ are negative, thus
    	\[ u(x,t) = e^{-(t+x/c)} + \frac{1}{2}\cos(x-ct) - \frac{1}{2}\cos(-x-ct) \]
    	Therefore
    	\[
    		\boxed{ u(x,t) = \begin{cases} \cos(x)\cos(ct) \hspace{2.06cm} x+ct<0  \\ e^{-(t+x/c)}+\sin(x)\sin(ct) \quad x+ct>0 \end{cases} }
    	\]
    \end{prob}
    
    
    \begin{prob}{12.5.1} $  $ \vspace{3mm} \\
    	$ \dfrac{\partial^2 u}{\partial t^2} = c^2 \dfrac{\partial^2 u}{\partial x^2} \, , \quad   \begin{cases} u(x,0) = f(x), \hspace{6mm} 0<x<L \vspace{1mm} \\ \dfrac{\partial u}{\partial t}(x,0) = g(x), \quad 0<x<L \vspace{1mm} \\ u(0,t) = u(L,t)=0 \end{cases} $ \\
    	
    	\begin{enumerate}[label=\alph*.)]
    		\item Using separation of variables, let $ u(x,t) = \phi(x) h(t) $. We then have
    			\[ \frac{h''}{c^2 h} = \frac{\phi''}{\phi} = -\lambda \]
    			Solving for $ \phi(x) $,
    			\begin{align*}
    				&\phi'' = -\lambda \phi \qquad \phi(0) = \phi(L) = 0 \\
    				&\phi(x) = \sin\left( \frac{n\pi x}{L} \right), \quad \lambda = \left( \frac{n\pi}{L} \right)^2
    			\end{align*}
    		Solving for $ h(t) $,
    			\begin{align*}
    				h'' &= -\lambda c^2 h \\
    					&= -c^2 \frac{n^2 \pi^2}{L^2} h \\
    				\Rightarrow h(t) &= A_n \cos\left( \frac{n\pi c t}{L} \right) + B_n \left( \frac{n\pi c t}{L} \right)
    			\end{align*}
    		We then have 
    		\[ u(x,t) = \sin\left( \frac{n\pi x}{L} \right) \left(  A_n \cos\left( \frac{n\pi c t}{L} \right) + B_n \left( \frac{n\pi c t}{L} \right)\right) \]
    		and by superposition,
    		\[ \boxed{ u(x,t) = \sum\limits_{n=1}^{\infty} \sin\left( \frac{n\pi x}{L} \right) \left(  A_n \cos\left( \frac{n\pi c t}{L} \right) + B_n \left( \frac{n\pi c t}{L} \right)\right) } \]
    		where 
    			\begin{subequations}
    				\begin{empheq}[box=\widefbox]{align*}
    					A_n &= \frac{2}{L} \int_{0}^{L} f(x) \sin\left( \frac{n\pi x}{L} \right) \, dx ,\\
    					B_n &= \frac{2}{n\pi c} \int_{0}^{L} g(x) \sin\left( \frac{n\pi x}{L} \right) \, dx
    				\end{empheq}
    			\end{subequations}
    			
    		\item If $ g(x) = 0 $, then $ B_n=0 $, yielding
    			\[ u(x,t) = \sum\limits_{n=1}^{\infty} A_n \sin\left( \frac{n\pi x}{L} \right) \cos\left( \frac{n\pi ct}{L} \right) \]
    			where $ f(x) = \sum_{n=1}^{\infty} A_n \sin(n\pi x/L), \, 0<x<L $. But this series does not $ f(x) $ outside of $ 0<x<L $. Instead we use the periodic extension of $ f(x) $:
    			\[ \bar{f}(x) = \sum\limits_{n=1}^{\infty} A_n \sin\left( \frac{n\pi x}{L} \right) \cos\left( \frac{n\pi ct}{L} \right) \]
    			where $ \bar{f}(x) $ denotes the periodic extension of $ f(x) $. Using the identity $ \sin\theta\cos\phi = \frac{1}{2}\sin(\theta+\phi) + \frac{1}{2} \sin(\theta-\phi) $, it follows that 
    			\[ u(x,t) = \frac{1}{2} \sum\limits_{n=1}^{\infty} \left[ A_n\sin\frac{n\pi}{L}(x+ct) + A_n\sin\frac{n\pi}{L}(x-ct) \right] \]
    			or
    			\[ \boxed{u(x,t) = \frac{1}{2}(\bar{f}(x+ct) + \bar{f}(x-ct)) } \]
    			
    		\item If $ f(x) = 0 $, then $ A_n = 0 $, yielding
    			\[ u(x,t) = \sum\limits_{n=1}^{\infty} B_n \sin\left( \frac{n\pi x}{L} \right) \left( \frac{n\pi ct}{L} \right) \]
    			following the same logic from part (b), we conclude that 
    			\[ \boxed{u(x,t) = \frac{1}{2}(\bar{f}(x+ct) + \bar{f}(x-ct)) } \]
    	\end{enumerate}
    \end{prob}
    
    
    
    
    
    
    
    
    
    
    
    
\end{document}