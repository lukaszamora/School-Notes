\documentclass{article}

\usepackage[margin=1in]{geometry}
\usepackage{amsmath,amsthm,tikz,fancyhdr,bm,enumitem,amssymb,empheq}
\usepackage{esint}

\theoremstyle{definition}

\newcommand*\widefbox[1]{\fbox{\hspace{2em}#1\hspace{2em}}}

\newtheorem{innercustomgeneric}{\customgenericname}
\providecommand{\customgenericname}{}
\newcommand{\newcustomtheorem}[2]{%
  \newenvironment{#1}[1]
  {%
   \renewcommand\customgenericname{#2}%
   \renewcommand\theinnercustomgeneric{##1}%
   \innercustomgeneric
  }
  {\endinnercustomgeneric}
}

\newcustomtheorem{prob}{Problem}
\newcustomtheorem{customlemma}{Lemma}

\pagestyle{fancy}
\fancyhf{}
\rhead{Lukas Zamora}
\chead{Bonus Homework}
\lhead{MATH 412}
\cfoot{\thepage}

\title{MATH 412 -- Bonus Homework}
\author{Lukas Zamora}
\date{December 5, 2018}

\setlength\parindent{0pt}


\begin{document}

    \maketitle
    
    \begin{prob}{1} $  $ \vspace{3mm} \\
        From the characteristic,
        \[
            \frac{dx}{dt} = 0
        \]
        Thus $u=c$ and 
        \begin{align*}
            \frac{dt}{1} = \frac{dx}{x} \quad \Rightarrow \quad \log(x) + \tilde{c} \quad &\Rightarrow x = \tilde{c} e^t \\
            &\Rightarrow \tilde{c} = xe^{-t}
        \end{align*}
        Hence our solution is of the form $u=f(xe^{-t})$. \\

        For $x<0$:
        \[
            4 = u(x,0) = f(x)
        \]
        For $3<x<4$:
        \[
            u(x,0) = 1 = f(x)
        \]
        For $0<x<3$:
        \[
            4-x = u(x,0) = f(x)
        \]
        For $x>4$:
        \[
            12 = u(x,0) = f(x)
        \]
        Thus 
        \[
            u(x,t) = \begin{cases} 4 & x<0 \\ 4-xe^{-t} & 0<x<3 \\ 1 & 3<x<4 \\ 12 &x>4   \end{cases}
        \]
        
    \end{prob}
    

    \begin{prob}{3} $  $ \vspace{3mm} \\
        Consider de'Alembert's formula,
        \[
            u(x,t) = \frac{1}{2}(f(x-ct) + f(x+ct)) + \frac{1}{2c}\int_{x-ct}^{x+ct} g(\bar{x}) \, d\bar{x}
        \]
        Here, $c=3$ and $g(x)=0$ from the initial condition. We then have
        \begin{align*}
            u(x,t) &= \frac{1}{2}(f(x-3t) + f(x+3t)) \\
                   &= \frac{1}{2} \left( e^{2(x-3t)} + e^{2(x+3t)} \right) \\
                   &= \frac{1}{2} e^{2x}\left( e^{-6t} + e^{6t} \right) \\
                   &= e^{2x} \cosh(6t)
        \end{align*}
        Checking our solution,
        \[
            u_{tt} = 36e^{2x}\cosh(6t) = 4e^{2x}\cosh(6t) = 9u_{xx}
        \]
        Thus
        \[
            u(x,t) = e^{2x}\cosh(6t)
        \]
    \end{prob}
    

\end{document}