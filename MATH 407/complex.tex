\documentclass{article}



\def\subj {MATH}
\def\npart {407}
\def\nterm {Fall}
\def\nyear {2018}
\def\ncourse {Complex Variables}

\RequirePackage{etex}
\makeatletter
\ifx \nauthor\undefined
  \def\nauthor{Lukas Zamora}
\else
\fi

\author{Notes taken by \nauthor}
\date{\nterm\ \nyear}

%\usepackage[margin=1in]{geometry}
\usepackage{alltt}
\usepackage{amsfonts}
\usepackage{amsmath}
\usepackage{amssymb}
\usepackage{bm}
\usepackage{amsthm}
\usepackage{booktabs}
\usepackage{caption}
\usepackage{enumitem}
\usepackage{fancyhdr}
\usepackage{graphicx}
\usepackage{mathdots}
\usepackage{mathtools}
\usepackage{microtype}
\usepackage{multirow}
\usepackage{pdflscape}
\usepackage{pgfplots}
\usepackage{siunitx}
\usepackage{textcomp}
\usepackage{slashed}
\usepackage{tabularx}
\usepackage{tikz}
\usepackage{tkz-euclide}
\usepackage[normalem]{ulem}
\usepackage[all]{xy}
\usepackage{imakeidx}
\usepackage{color}
\usepackage{parcolumns}

% Table stuff
\usepackage{array}
\usepackage{makecell}
\renewcommand\theadalign{bl}
\renewcommand\theadfont{\bfseries}
\renewcommand\theadgape{\Gape[2pt]}
\renewcommand\cellgape{\Gape[2pt]}


\definecolor{pblue}{rgb}{0.13,0.13,1}
\definecolor{pgreen}{rgb}{0,0.5,0}
\definecolor{pred}{rgb}{0.9,0,0}
\definecolor{pgrey}{rgb}{0.46,0.45,0.48}

\usepackage{listings}
\lstset{language=C++,
	showspaces=false,
	showtabs=false,
	breaklines=true,
	showstringspaces=false,
	breakatwhitespace=true,
	commentstyle=\color{pgreen},
	keywordstyle=\color{pblue},
	stringstyle=\color{pred},
	basicstyle=\fontsize{10}{10}\selectfont\ttfamily,
	moredelim=[il][\textcolor{pgrey}]{$$},
	moredelim=[is][\textcolor{pgrey}]{\%\%}{\%\%}
}


\makeindex[intoc, title=Index]
\indexsetup{othercode={\lhead{\emph{Index}}}}

\ifx \nextra \undefined
  \usepackage[pdftex,
    hidelinks,
    pdfauthor={Lukas Zamora},
    pdfsubject={Texas A\&M Maths Notes: \npart\ - \ncourse},
    pdftitle={Part \npart\ - \ncourse},
  pdfkeywords={A\&M Mathematics Maths Math \npart\ \nterm\ \nyear\ \ncourse}]{hyperref}
  \title{\subj\ \npart\ --- \ncourse}
\else
  \usepackage[pdftex,
    hidelinks,
    pdfauthor={Lukas Zamora},
    pdfsubject={Texas A\&M Math Notes: \npart\ - \ncourse\ (\nextra)},
    pdftitle={\npart\ - \ncourse\ (\nextra)},
  pdfkeywords={A\&M Mathematics Maths Math \npart\ \nterm\ \nyear\ \ncourse\ \nextra}]{hyperref}

  \title{\subj\ \npart\ --- \ncourse \\ {\Large \nextra}}
  \renewcommand\printindex{}
\fi

\pgfplotsset{compat=1.12}

\pagestyle{fancyplain}
\ifx \ncoursehead \undefined
\def\ncoursehead{\ncourse}
\fi

\lhead{\emph{\nouppercase{\leftmark}}}
\ifx \nextra \undefined
  \rhead{
    \ifnum\thepage=1
    \else
       \subj \hspace*{0.4mm} \npart
    \fi}
\else
  \rhead{
    \ifnum\thepage=1
    \else
      \npart\ \ncoursehead \ (\nextra)
    \fi}
\fi
\usetikzlibrary{arrows.meta}
\usetikzlibrary{decorations.markings}
\usetikzlibrary{decorations.pathmorphing}
\usetikzlibrary{positioning}
\usetikzlibrary{fadings}
\usetikzlibrary{intersections}
\usetikzlibrary{cd}
\usetikzlibrary{shapes}
\usetikzlibrary{hobby}
\usetikzlibrary{calc}


\newcommand*{\Cdot}{{\raisebox{-0.25ex}{\scalebox{1.5}{$\cdot$}}}}
\newcommand {\pd}[2][ ]{
  \ifx #1 { }
    \frac{\partial}{\partial #2}
  \else
    \frac{\partial^{#1}}{\partial #2^{#1}}
  \fi
}
\ifx \nhtml \undefined
\else
  \renewcommand\printindex{}
  \DisableLigatures[f]{family = *}
  \let\Contentsline\contentsline
  \renewcommand\contentsline[3]{\Contentsline{#1}{#2}{}}
  \renewcommand{\@dotsep}{10000}
  \newlength\currentparindent
  \setlength\currentparindent\parindent

  \newcommand\@minipagerestore{\setlength{\parindent}{\currentparindent}}
  \usepackage[active,tightpage,pdftex]{preview}
  \renewcommand{\PreviewBorder}{0.1cm}

  \newenvironment{stretchpage}%
  {\begin{preview}\begin{minipage}{\hsize}}%
    {\end{minipage}\end{preview}}
  \AtBeginDocument{\begin{stretchpage}}
  \AtEndDocument{\end{stretchpage}}

  \newcommand{\@@newpage}{\end{stretchpage}\begin{stretchpage}}

  \let\@real@section\section
  \renewcommand{\section}{\@@newpage\@real@section}
  \let\@real@subsection\subsection
  \renewcommand{\subsection}{\@ifstar{\@real@subsection*}{\@@newpage\@real@subsection}}
\fi
\ifx \ntrim \undefined
\else
  \usepackage{geometry}
  \geometry{
    papersize={379pt, 699pt},
    textwidth=345pt,
    textheight=596pt,
    left=17pt,
    top=54pt,
    right=17pt
  }
\fi


% Contour Lines Tikz


\ifx \nisofficial \undefined
\let\@real@maketitle\maketitle
%\renewcommand{\maketitle}{\@real@maketitle\begin{center}\begin{minipage}[c]{0.9\textwidth}\centering\footnotesize These notes are not endorsed by the lecturers, and I have modified them (often significantly) after lectures. They are nowhere near accurate representations of what was actually lectured, and in particular, all errors are almost surely mine.\end{minipage}\end{center}}
\else
\fi

% Theorems
\theoremstyle{definition}
\newtheorem*{aim}{Aim}
\newtheorem*{axiom}{Axiom}
\newtheorem*{claim}{Claim}
\newtheorem*{cor}{Corollary}
\newtheorem*{conjecture}{Conjecture}
\newtheorem*{define}{Definition}
\newtheorem*{eg}{Example}
\newtheorem*{ex}{Exercise}
\newtheorem*{fact}{Fact}
\newtheorem*{law}{Law}
\newtheorem*{lemma}{Lemma}
\newtheorem*{notation}{Notation}
\newtheorem*{prop}{Proposition}
\newtheorem*{question}{Question}
\newtheorem*{problem}{Problem}
\newtheorem*{rrule}{Rule}
\newtheorem*{thm}{Theorem}
\newtheorem*{assumption}{Assumption}

\newtheorem*{remark}{Remark}
\newtheorem*{warning}{Warning}
\newtheorem*{exercise}{Exercise}

\newtheorem{nthm}{Theorem}[section]
\newtheorem{nlemma}[nthm]{Lemma}
\newtheorem{nprop}[nthm]{Proposition}
\newtheorem{ncor}[nthm]{Corollary}


\renewcommand{\labelitemi}{--}
\renewcommand{\labelitemii}{$\circ$}
\renewcommand{\labelenumi}{(\roman{*})}

\let\stdsection\section
\renewcommand\section{\newpage\stdsection}

% Strike through
\def\st{\bgroup \ULdepth=-.55ex \ULset}


%%%%%%%%%%%%%%%%%%%%%%%%%
%%%%% Maths Symbols %%%%%
%%%%%%%%%%%%%%%%%%%%%%%%%

% Matrix groups
\newcommand{\GL}{\mathrm{GL}}
\newcommand{\Or}{\mathrm{O}}
\newcommand{\PGL}{\mathrm{PGL}}
\newcommand{\PSL}{\mathrm{PSL}}
\newcommand{\PSO}{\mathrm{PSO}}
\newcommand{\PSU}{\mathrm{PSU}}
\newcommand{\SL}{\mathrm{SL}}
\newcommand{\SO}{\mathrm{SO}}
\newcommand{\Spin}{\mathrm{Spin}}
\newcommand{\Sp}{\mathrm{Sp}}
\newcommand{\SU}{\mathrm{SU}}
\newcommand{\U}{\mathrm{U}}
\newcommand{\Mat}{\mathrm{Mat}}

% Matrix algebras
\newcommand{\gl}{\mathfrak{gl}}
\newcommand{\ort}{\mathfrak{o}}
\newcommand{\so}{\mathfrak{so}}
\newcommand{\su}{\mathfrak{su}}
\newcommand{\uu}{\mathfrak{u}}
\renewcommand{\sl}{\mathfrak{sl}}

% Special sets
\newcommand{\C}{\mathbb{C}}
\newcommand{\CP}{\mathbb{CP}}
\newcommand{\GG}{\mathbb{G}}
\newcommand{\N}{\mathbb{N}}
\newcommand{\Q}{\mathbb{Q}}
\newcommand{\R}{\mathbb{R}}
\newcommand{\RP}{\mathbb{RP}}
\newcommand{\T}{\mathbb{T}}
\newcommand{\Z}{\mathbb{Z}}
\renewcommand{\H}{\mathbb{H}}

% Brackets
\newcommand{\abs}[1]{\left\lvert #1\right\rvert}
\newcommand{\bket}[1]{\left\lvert #1\right\rangle}
\newcommand{\brak}[1]{\left\langle #1 \right\rvert}
\newcommand{\braket}[2]{\left\langle #1\middle\vert #2 \right\rangle}
\newcommand{\bra}{\langle}
\newcommand{\ket}{\rangle}
\newcommand{\norm}[1]{\left\lVert #1\right\rVert}
\newcommand{\normalorder}[1]{\mathop{:}\nolimits\!#1\!\mathop{:}\nolimits}
\newcommand{\tv}[1]{|#1|}
\renewcommand{\vec}[1]{\boldsymbol{\mathbf{#1}}}

% not-math
\newcommand{\bolds}[1]{{\bfseries #1}}
\newcommand{\cat}[1]{\mathsf{#1}}
\newcommand{\ph}{\,\cdot\,}
\newcommand{\term}[1]{\emph{#1}\index{#1}}
\newcommand{\phantomeq}{\hphantom{{}={}}}
% Probability
\DeclareMathOperator{\Bernoulli}{Bernoulli}
\DeclareMathOperator{\betaD}{beta}
\DeclareMathOperator{\bias}{bias}
\DeclareMathOperator{\binomial}{binomial}
\DeclareMathOperator{\corr}{corr}
\DeclareMathOperator{\cov}{cov}
\DeclareMathOperator{\gammaD}{gamma}
\DeclareMathOperator{\mse}{mse}
\DeclareMathOperator{\multinomial}{multinomial}
\DeclareMathOperator{\Poisson}{Poisson}
\DeclareMathOperator{\var}{var}
\newcommand{\E}{\mathbb{E}}
\newcommand{\Prob}{\mathbb{P}}

% Algebra
\DeclareMathOperator{\adj}{adj}
\DeclareMathOperator{\Ann}{Ann}
\DeclareMathOperator{\Aut}{Aut}
\DeclareMathOperator{\Char}{char}
\DeclareMathOperator{\disc}{disc}
\DeclareMathOperator{\dom}{dom}
\DeclareMathOperator{\fix}{fix}
\DeclareMathOperator{\Hom}{Hom}
\DeclareMathOperator{\id}{id}
\DeclareMathOperator{\image}{image}
\DeclareMathOperator{\im}{im}
\DeclareMathOperator{\re}{re}
\DeclareMathOperator{\tr}{tr}
\DeclareMathOperator{\Tr}{Tr}
\newcommand{\Bilin}{\mathrm{Bilin}}
\newcommand{\Frob}{\mathrm{Frob}}

% Others
\newcommand\ad{\mathrm{ad}}
\newcommand\Art{\mathrm{Art}}
\newcommand{\B}{\mathcal{B}}
\newcommand{\cU}{\mathcal{U}}
\newcommand{\Der}{\mathrm{Der}}
\newcommand{\D}{\mathrm{D}}
\newcommand{\dR}{\mathrm{dR}}
\newcommand{\exterior}{\mathchoice{{\textstyle\bigwedge}}{{\bigwedge}}{{\textstyle\wedge}}{{\scriptstyle\wedge}}}
\newcommand{\F}{\mathbb{F}}
\newcommand{\G}{\mathcal{G}}
\newcommand{\Gr}{\mathrm{Gr}}
\newcommand{\haut}{\mathrm{ht}}
\newcommand{\Hol}{\mathrm{Hol}}
\newcommand{\hol}{\mathfrak{hol}}
\newcommand{\Id}{\mathrm{Id}}
\newcommand{\lie}[1]{\mathfrak{#1}}
\newcommand{\op}{\mathrm{op}}
\newcommand{\Oc}{\mathcal{O}}
\newcommand{\pr}{\mathrm{pr}}
\newcommand{\Ps}{\mathcal{P}}
\newcommand{\pt}{\mathrm{pt}}
\newcommand{\qeq}{\mathrel{``{=}"}}
\newcommand{\Rs}{\mathcal{R}}
\newcommand{\Vect}{\mathrm{Vect}}
\newcommand{\wsto}{\stackrel{\mathrm{w}^*}{\to}}
\newcommand{\wt}{\mathrm{wt}}
\newcommand{\wto}{\stackrel{\mathrm{w}}{\to}}
\renewcommand{\d}{\mathrm{d}}
\renewcommand{\P}{\mathbb{P}}
%\renewcommand{\F}{\mathcal{F}}


\let\Im\relax
\let\Re\relax

\DeclareMathOperator{\area}{area}
\DeclareMathOperator{\card}{card}
\DeclareMathOperator{\ccl}{ccl}
\DeclareMathOperator{\ch}{ch}
\DeclareMathOperator{\cl}{cl}
\DeclareMathOperator{\cls}{\overline{\mathrm{span}}}
\DeclareMathOperator{\coker}{coker}
\DeclareMathOperator{\conv}{conv}
\DeclareMathOperator{\cosec}{cosec}
\DeclareMathOperator{\cosech}{cosech}
\DeclareMathOperator{\covol}{covol}
\DeclareMathOperator{\diag}{diag}
\DeclareMathOperator{\diam}{diam}
\DeclareMathOperator{\Diff}{Diff}
\DeclareMathOperator{\End}{End}
\DeclareMathOperator{\energy}{energy}
\DeclareMathOperator{\erfc}{erfc}
\DeclareMathOperator{\erf}{erf}
\DeclareMathOperator*{\esssup}{ess\,sup}
\DeclareMathOperator{\ev}{ev}
\DeclareMathOperator{\Ext}{Ext}
\DeclareMathOperator{\fst}{fst}
\DeclareMathOperator{\Fit}{Fit}
\DeclareMathOperator{\Frac}{Frac}
\DeclareMathOperator{\Gal}{Gal}
\DeclareMathOperator{\gr}{gr}
\DeclareMathOperator{\hcf}{hcf}
\DeclareMathOperator{\Im}{Im}
\DeclareMathOperator{\Ind}{Ind}
\DeclareMathOperator{\Int}{Int}
\DeclareMathOperator{\Isom}{Isom}
\DeclareMathOperator{\lcm}{lcm}
\DeclareMathOperator{\length}{length}
\DeclareMathOperator{\Lie}{Lie}
\DeclareMathOperator{\like}{like}
\DeclareMathOperator{\Lk}{Lk}
\DeclareMathOperator{\Maps}{Maps}
\DeclareMathOperator{\orb}{orb}
\DeclareMathOperator{\ord}{ord}
\DeclareMathOperator{\otp}{otp}
\DeclareMathOperator{\poly}{poly}
\DeclareMathOperator{\rank}{rank}
\DeclareMathOperator{\rel}{rel}
\DeclareMathOperator{\Rad}{Rad}
\DeclareMathOperator{\Re}{Re}
\DeclareMathOperator*{\res}{res}
\DeclareMathOperator{\Res}{Res}
\DeclareMathOperator{\Ric}{Ric}
\DeclareMathOperator{\rk}{rk}
\DeclareMathOperator{\Rees}{Rees}
\DeclareMathOperator{\Root}{Root}
\DeclareMathOperator{\sech}{sech}
\DeclareMathOperator{\sgn}{sgn}
\DeclareMathOperator{\snd}{snd}
\DeclareMathOperator{\Spec}{Spec}
\DeclareMathOperator{\spn}{span}
\DeclareMathOperator{\stab}{stab}
\DeclareMathOperator{\St}{St}
\DeclareMathOperator{\supp}{supp}
\DeclareMathOperator{\Syl}{Syl}
\DeclareMathOperator{\Sym}{Sym}
\DeclareMathOperator{\vol}{vol}

\newcommand{\ihat}{\boldsymbol{\hat{\textbf{\i}}}}
\newcommand{\jhat}{\boldsymbol{\hat{\textbf{\j}}}}
\newcommand{\khat}{\boldsymbol{\hat{\textbf{k}}}}


\tikzset{circ/.style = {fill, circle, inner sep = 0, minimum size = 3}}
\tikzset{scirc/.style = {fill, circle, inner sep = 0, minimum size = 1.5}}
\tikzset{mstate/.style={circle, draw, blue, text=black, minimum width=0.7cm}}

\tikzset{eqpic/.style={baseline={([yshift=-.5ex]current bounding box.center)}}}
\tikzset{commutative diagrams/.cd,cdmap/.style={/tikz/column 1/.append style={anchor=base east},/tikz/column 2/.append style={anchor=base west},row sep=tiny}}

\definecolor{mblue}{rgb}{0.2, 0.3, 0.8}
\definecolor{morange}{rgb}{1, 0.5, 0}
\definecolor{mgreen}{rgb}{0.1, 0.4, 0.2}
\definecolor{mred}{rgb}{0.5, 0, 0}

\def\drawcirculararc(#1,#2)(#3,#4)(#5,#6){%
    \pgfmathsetmacro\cA{(#1*#1+#2*#2-#3*#3-#4*#4)/2}%
    \pgfmathsetmacro\cB{(#1*#1+#2*#2-#5*#5-#6*#6)/2}%
    \pgfmathsetmacro\cy{(\cB*(#1-#3)-\cA*(#1-#5))/%
                        ((#2-#6)*(#1-#3)-(#2-#4)*(#1-#5))}%
    \pgfmathsetmacro\cx{(\cA-\cy*(#2-#4))/(#1-#3)}%
    \pgfmathsetmacro\cr{sqrt((#1-\cx)*(#1-\cx)+(#2-\cy)*(#2-\cy))}%
    \pgfmathsetmacro\cA{atan2(#2-\cy,#1-\cx)}%
    \pgfmathsetmacro\cB{atan2(#6-\cy,#5-\cx)}%
    \pgfmathparse{\cB<\cA}%
    \ifnum\pgfmathresult=1
        \pgfmathsetmacro\cB{\cB+360}%
    \fi
    \draw (#1,#2) arc (\cA:\cB:\cr);%
}
\newcommand\getCoord[3]{\newdimen{#1}\newdimen{#2}\pgfextractx{#1}{\pgfpointanchor{#3}{center}}\pgfextracty{#2}{\pgfpointanchor{#3}{center}}}

\newcommand\qedshift{\vspace{-17pt}}
\newcommand\fakeqed{\pushQED{\qed}\qedhere}

\def\Xint#1{\mathchoice
   {\XXint\displaystyle\textstyle{#1}}%
   {\XXint\textstyle\scriptstyle{#1}}%
   {\XXint\scriptstyle\scriptscriptstyle{#1}}%
   {\XXint\scriptscriptstyle\scriptscriptstyle{#1}}%
   \!\int}
\def\XXint#1#2#3{{\setbox0=\hbox{$#1{#2#3}{\int}$}
     \vcenter{\hbox{$#2#3$}}\kern-.5\wd0}}
\def\ddashint{\Xint=}
\def\dashint{\Xint-}

\newcommand\separator{{\centering\rule{2cm}{0.2pt}\vspace{2pt}\par}}

\newenvironment{own}{\color{gray!70!black}}{}

\newcommand\makecenter[1]{\raisebox{-0.5\height}{#1}}

\mathchardef\mdash="2D

\newenvironment{significant}{\begin{center}\begin{minipage}{0.9\textwidth}\centering\em}{\end{minipage}\end{center}}
\DeclareRobustCommand{\rvdots}{%
  \vbox{
    \baselineskip4\p@\lineskiplimit\z@
    \kern-\p@
    \hbox{.}\hbox{.}\hbox{.}
  }}
\DeclareRobustCommand\tph[3]{{\texorpdfstring{#1}{#2}}}
\makeatother

\begin{document}
\maketitle
{\small
  \noindent\textbf{The Complex Number Plane}\\
  Introduction to complex numbers, the complex plane, point sets in the plane, stereographic projection; the extended complex plane, curves and regions.\hspace*{\fill} [1]

  \vspace{10pt}
  \noindent\textbf{Functions of a Complex Variable}\\
  Functions and limits, differentiability and analyticity, the Cauchy-Riemann conditions, linear fractional transformations, transcendental functions, Riemann surfaces.\hspace*{\fill} [2]

  \vspace{10pt}
  \noindent\textbf{Integration in the Complex Plane}\\
  Line integrals, Cauchy's theorem, Cauchy formulas, Maximum Modulus Principle.\hspace*{\fill} [3]

  \vspace{10pt}
  \noindent\textbf{Sequences and Series}\\
  Sequences of complex numbers; functions, infinite series, power series, analytic continuation, Laurent series, Double series, infinite products, improper integrals, the Gamma function.\hspace*{\fill} [4]

  \vspace{10pt}
  \noindent\textbf{Residue Calculus}\\
  The Residue theorem, evaluation of real integrals, the principle of the argument, meromorphic functions, entire functions.\hspace*{\fill} [5]
  }

\tableofcontents

\setcounter{section}{0}
\section{The Complex Numbers}

	\textbf{Question}: Does $x^2+1=0$ have any solutions?
	\begin{itemize}
		\item No: If we look for real solutions 
		\item Yes: If we have a more general notion of numbers
	\end{itemize}	

	\noindent We introduce $i$ so that $i^2 = -1$.

	\begin{defi}
		$\C$ is the set of complex numbers formed as $z=x+iy$, $x,y \in \R$. The \textit{real part}, $x$, is written as $\Re(z)$, and the \textit{imaginary part}, $y$, is written as $\Im(z)$.
	\end{defi}

	\begin{note}
		We can identify $\C$ with points in $\R^2$ by the correspondence $(x+iy) \leftrightarrow (x,y)$.
	\end{note}

	
	\begin{defi}[Addition/Subtraction]
		We add/subtract complex numbers by their real and imaginary components respectively.
		\[
			(a+ib) \pm (c+id) = (a\pm c) + i(b \pm d)
		\]
	\end{defi}

	\begin{defi}
		The \textit{Modulus} (absolute value) of $z=x+iy$ is the length of the vector $(x,y)$ which is $\sqrt{x^2+y^2}$, and is denoted as $|z|$.
	\end{defi}

	\begin{defi}[Triangle Inequality]

		For vectors, we have the triangle inequality
		\[
			|\vec{u} + \vec{v}| \leq |\vec{u}| + |\vec{v}|
		\]
		For $\C$, this translates to
		\[
			|z_1 + z_2| \leq |z_1| + |z_2|
		\]
		There is also a useful variant: \\
	
			Apply to $z=(z-w) + w$, where $z,w \in \C$.
			\begin{align*}
				|z| &\leq |z-w| + |w| \\
				|z-w| &\geq |z| - |w| \\
				|w| - |z| &\leq |w-z| = |z-w| \\
				\Rightarrow |z-w| &\geq \big| |z| - |w| \big| 
		\end{align*}
	\end{defi}

	\begin{defi}[Multiplication]
		For multiplying two complex numbers, we expand each binomial and collect the real and imaginary parts respectively.
		\[
			(a+ib)(c+id) = (ac-bd) + i(bc + ad)
		\]
	\end{defi}

	The usual algebraic rules still apply for complex numbers.
	\begin{alignat*}{2}
		(z_1 z_2) z_3 &= z_1(z_2 z_3) \qquad  &&\text{associative} \\
		z_1 z_2 &= z_2 z_1 \qquad &&\text{commutative} \\
		z_1(z_2 + z_3) &= z_1 z_2 + z_1 z_3 \qquad &&\text{distributive}
	\end{alignat*}

	\begin{eg}[Proof of commutative] $ $
		\begin{proof}
			\begin{align*}
				(a+ib)(c+id) &= (ac-bd) + i(bc + ad) \\
				(c+id)(a+ib) &= (ca-db) + i(cb + da)
			\end{align*}
		\end{proof}
	\end{eg}

	\begin{defi}[Complex conjugate]
		If $z=x+iy$, then its complex conjugate, denoted as $\bar{z}$, is $\bar{z} = x-iy$.
	\end{defi}

	\begin{eg}
		If we compute $z\bar{z}$, we get
		\begin{align*}
			(z+iy)(z-iy) &= x^2 + y^2 + i(yx-xy) \\
			&= x^2 + y^2 \\
			&= |z|^2
		\end{align*}
		i.e $z\bar{z} = |z|^2$.
	\end{eg}
	Some other useful properties:
	\begin{align*}
		\overline{(z+w)} &= \bar{z} + \bar{w} \\
		\overline{zw} &= \bar{z} \bar{w} \\
		|\bar{z}| &= |z|
	\end{align*}

	\begin{note}
		$z+\bar{z}/2 = x$ , $z-\bar{z}/2i = y$.
	\end{note}

	\begin{defi}[Inverses]
		If $z\neq 0$, then $1/z$ exists as a complex number. We have
		\[
			z\bar{z} = |z^2|
		\]
		or
		\[
			\frac{1}{z} = \frac{\bar{z}}{|z|^2}
		\]
	\end{defi}

	\begin{defi}[Division]
		It follows from the definition of a complex inverse,
		\[
			\frac{z}{w} = \frac{z\bar{w}}{w\bar{w}} = \frac{z\bar{w}}{|w|^2}
		\]
	\end{defi}

	\begin{thm}[Fundamental Theorem of Algebra]
		Every polynomial can be expressed as a product of linear factors \underline{over $\C$}. This is equivalent to the statement that every polynomial equation $p(z)=0$ has a solution.
	\end{thm}

	\begin{proof}
		Suppose a polynomial $p(z)$ factors as $(z-z_{1})(z-z_{2}) \dots (z-z_{n})$, then we have solutions $z_{1}, \dots ,z_{n}$ for $p(z)=0$.\\

		Suppose	all equations $p(z)=0$ have solutions. We want all $p(z)$'s to factor. Do induction on the degree of $p(z)$.\\

		If $\degree p(z) = 1$, then $p(z) = (z-z_{1})$, so it factors. Suppose polynomials of $\degree n-1$ factor, and consider an $n$ degree polynomial $p(x)$, so $p(z)=0$ has a solution. So there is $z_{1}$ with $p(z_{1})=0$. We can write
		\[
			p(z) = (z-z_1)q(z)
		\]
		with $\degree q(z) = n-1$. Then $q(z)$ factors as $(z-z_{1})(z-z_{2}) \dots (z-z_{n})$, so $p(z)$ factors.
	\end{proof}

	\begin{remark}
		If the Fundamental Theorem is true, then for any complex number $w$, the equation $z^n-w=0$ has a solution, i.e all complex numbers have $n^{th}$ roots.
	\end{remark}

	\subsection*{Polar Form}

	Sometimes, working in a different coordinate system helps simplify things. We introduce the polar coordinate system, where $r$ is the distance from the point to the origin, and $\theta$ is the angle from the point to the $x$-axis.

	\begin{center}
		\begin{tikzpicture}
			\draw (-0.5,0) -- (3,0);
			\draw (0,-0.5) -- (0,2);

			\draw[thick] (0,0) -- (2,1);
			\node at (1,1) [below] {$r$};
			\draw (0.8,0) to[bend right] (0.6,0.315);
			\node at (1,0) [above] {$\theta$};
			\node[circle,fill=black,inner sep=0pt,minimum size=3pt,label=right:{$(x,y)$}] at (2,1) {};
		\end{tikzpicture}
	\end{center}

	To convert from Cartesian to polar and back, we use the following formulas
	\[
		x = r\cos\theta \qquad y = r\sin\theta
	\]
	We also adopt the convention that $-\pi \leq \theta \leq \pi$.

	\begin{defi}[Argument]
		We define the \textit{argument} of $z$ to be all $\theta$'s so that $z = r\cos\theta + i r\sin\theta$. We write this as $\arg z$. There is always one value in $(-\pi,\pi]$ and this the \textit{principal value}, written as $\Arg z$. We then have that 
		\[
			\arg z = \{ \Arg z + 2\pi n \, | \, n \in \Z \}
		\]
	\end{defi}

	\begin{eg} $ $
		\begin{enumerate}[label=\alph*.)]
			\item $\Arg(-1) = \pi$, $\arg(-1) = \{ (2n+1)\pi \, | \, n \in \Z \}$.
			\item $\Arg(i) = \pi/2$, $\arg(i) = \{ (2n+\frac{1}{2})\pi \, | \, n \in \Z \}$.
		\end{enumerate}
	\end{eg}

	In Calculus, we have the following Taylor Series
	\begin{align*}
		e^x &= 1 + x + \frac{x^2}{2!} + \frac{x^3}{3!} + \dots \\
		\sin x &= x - \frac{x^3}{3!} + \frac{x^5}{5!} + \dots \\
		\cos x &= 1 - \frac{x^2}{2!} + \frac{x^4}{4!} + \dots
	\end{align*}
	This suggests
	\begin{align*}
		e^{i\theta} &= 1 + i\theta + \frac{i^2\theta^2}{2!} + \frac{i^3\theta^3}{3!} + \frac{i^4\theta^4}{4!} + \dots \\
		&= \left( 1 - \frac{\theta^2}{2!} + \frac{\theta^4}{4!} - \dots \right) + i\left( \theta - \frac{\theta^3}{3!} + \frac{\theta^5}{5!} - \dots \right) \\
		&= \cos\theta + i\sin\theta
	\end{align*}

	Since any $z$ can be written in polar as $z=r\cos\theta + ir\sin\theta$, we can write this as $z=re^{i\theta}$.



\end{document}
