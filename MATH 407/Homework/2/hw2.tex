\documentclass{article}

\usepackage[margin=1in]{geometry}
\usepackage{amsmath,amsthm,tikz,fancyhdr,bm,enumitem,amssymb}
\usepackage{empheq}

\newcommand*\widefbox[1]{\fbox{\hspace{2em}#1\hspace{2em}}}

\theoremstyle{definition}

\newtheorem{innercustomgeneric}{\customgenericname}
\providecommand{\customgenericname}{}
\newcommand{\newcustomtheorem}[2]{%
  \newenvironment{#1}[1]
  {%
   \renewcommand\customgenericname{#2}%
   \renewcommand\theinnercustomgeneric{##1}%
   \innercustomgeneric
  }
  {\endinnercustomgeneric}
}

\newcustomtheorem{prob}{Problem}
\newcustomtheorem{customlemma}{Lemma}

\pagestyle{fancy}
\fancyhf{}
\rhead{Lukas Zamora}
\chead{Homework 2}
\lhead{MATH 412}
\cfoot{\thepage}

\title{MATH 407 - Homework 2}
\author{Lukas Zamora}
\date{September 13, 2018}

\setlength\parindent{0pt}



\begin{document}

    \maketitle
    
    \begin{prob}{2.53c} $  $ \vspace{1mm} \\
    	Let $ z = x+iy $, then
    	\begin{align*}
    		f(z) & = \frac{1-z}{1+z} \\
    		     & = \frac{1-(x+iy)}{1+(x+iy)} \\
    		     & = \frac{(1-x) + iy}{(1+x) + iy} \cdot \frac{(1+x - iy)}{(1+x) - iy} \\
    		     & = \frac{x^2 - 1 -i (y+yx) + i(y+yx) - y^2 - i2y}{(x+1)^2 - i(y+yx) + i(y+yx) - y^2} \\
    		     & = \frac{(1 - x^2 - y^2) - i2y}{(1+x^2)^2 + y^2} \\
    		     & = \frac{1-x^2-y^2}{(1+x^2)^2 + y^2} - i \frac{2y}{(1+x^2)^2 + y^2} \\
    		     & \therefore \boxed{u(x,y) = \frac{1-x^2-y^2}{(1+x^2)^2 + y^2} \; , \; v(x,y) = \frac{-2y}{(1+x^2)^2 + y^2} }
    	\end{align*}
   	\end{prob}
   
   	\begin{prob}{2.58a} $  $
   		\begin{proof}
   			Let $ z_1 = a+ib, z_2 = c+id $. By the addition property, we have 
   			\begin{align*}
   				e^{z_1 + (-z_2)} & = e^{a+ib + (-c-id)} \\
   								 & = e^{a+ib} e^{-c-id}
   			\end{align*}
   			Using the reciprocal property, we know that $ e^{-c-id} = 1/e^{c+id} $. Combining these, we obtain
   			\[
   				e^{a+ib} \left( \frac{1}{e^{c+id}} \right) = \frac{e^{a+ib}}{e^{c+id}}
   			\]
   		\end{proof}
   	\end{prob}
   
   \begin{prob}{2.61b} $  $ \vspace{1mm} \\ 
		 In polar form, $ i = e^{\pi i/2} $. We then have 
		 \begin{align*}
		 	e^{4z} & = e^{\pi i/2} \\
		 	e^{4z} & = e^{(\pi/2 + 2k\pi)i} \\
		 		4z & = \left( \frac{\pi}{2} + 2k\pi \right)i \\
		 		   & \therefore \boxed{z = \left( \frac{\pi}{8} + \frac{\pi}{2}k \right)i \; , \; k = \pm 1, \pm 2, \dots }
		 \end{align*}
   \end{prob}

	\begin{prob}{2.62b} $  $
		\begin{proof}
			We know that
			\[
				\cos(z) = \frac{e^{iz} + e^{-iz}}{2} \; , \; \sin(z) = \frac{e^{iz} - e^{-iz}}{2i}
			\]
			We then have 
			\begin{align*}
				\cos^2(z) - \sin^2(z) &= \left( \frac{e^{iz} + e^{-iz}}{2} \right)^2 + \left( \frac{e^{iz} - e^{-iz}}{2i} \right)^2 \\
				&= \frac{1}{4} (e^{iz} + e^{-iz})^2 + \frac{1}{4}(e^{iz}-e^{-iz})^2 \\
				&= \frac{1}{4} ( e^{2iz} + 2 + e^{-2iz} + e^{2iz} - 2 + e^{-2iz} ) \\
				&= \frac{1}{2} (e^{2iz} + e^{-2iz} ) \\
				&= \cos(2z)
			\end{align*}
		\end{proof}
	\end{prob}

	\begin{prob}{2.68d} $  $ \vspace{1mm} \\
		Let $ z = x+iy $. We then have
		\[
			e^{2z} = e^{2(x+iy)} = e^{2x+i2y} = e^{2x} e^{i2y} = e^{2x} ( \cos(2y) + i\sin(2y) )
		\]
		and
		\[
			z^2 = (x+iy)^2 = (x^2-y^2) + i2xy
		\]
		Combining these, we get
		\begin{align*}
			z^2 e^{2z} &= ( (x^2-y^2) + i2xy )( e^{2x}\cos(2y) + ie^{2x}\sin(2y) ) \\
			&= e^{2x} x^2 \cos(2y) + i x^2 e^{2x} \sin(2y) - y^2 e^{2x} \cos(2y) - i y^2 e^{2x} \sin(2y) + i 2xye^{2x}\cos(2y) - 2xye^{2x}\sin(2y) \\
			&= e^{2x} ( x^2\cos(2y) - y^2\cos(2y) - 2xy\sin(2y) ) + ie^{2x} ( x^2\sin(2y) - y^2\sin(2y) + 2xy\cos(2y) ) \\
			&= e^{2x} ( (x^2-y^2) \cos(2y) - 2xy\sin(2y) ) + i e^{2x}( 2xy\cos(2y) + (x^2-y^2)\sin(2y) )  
		\end{align*} 
		\begin{empheq}[box=\fbox]{align}
			u(x,y) & = e^{2x} ( (x^2-y^2) \cos(2y) - 2xy\sin(2y) ) \nonumber \\
			v(x,y) & = e^{2x}( 2xy\cos(2y) + (x^2-y^2)\sin(2y) ) \nonumber
		\end{empheq}
	\end{prob}
    
    
\end{document}