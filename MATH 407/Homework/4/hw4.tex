\documentclass{article}

\usepackage[margin=1in]{geometry}
\usepackage{amsmath,amsthm,tikz,fancyhdr,bm,enumitem,amssymb,mathtools}
\usepackage{empheq}

\newcommand*\widefbox[1]{\fbox{\hspace{2em}#1\hspace{2em}}}

% L-Hospitals rule thing
\newcommand\myeq{\stackrel{\mathclap{\normalfont\mbox{\normalfont\tiny L.H}}}{=}}


\theoremstyle{definition}

\newtheorem{innercustomgeneric}{\customgenericname}
\providecommand{\customgenericname}{}
\newcommand{\newcustomtheorem}[2]{%
  \newenvironment{#1}[1]
  {%
   \renewcommand\customgenericname{#2}%
   \renewcommand\theinnercustomgeneric{##1}%
   \innercustomgeneric
  }
  {\endinnercustomgeneric}
}

\newcustomtheorem{prob}{Problem}
\newcustomtheorem{customlemma}{Lemma}

\pagestyle{fancy}
\fancyhf{}
\rhead{Lukas Zamora}
\chead{Homework 4}
\lhead{MATH 412}
\cfoot{\thepage}

\title{MATH 407 - Homework 4}
\author{Lukas Zamora}
\date{September 27, 2018}

\setlength\parindent{0pt}



\begin{document}

    \maketitle
    
    \begin{prob}{2.102} $  $ \\
    	First note that $ f(z) = \dfrac{z^2+4}{z-2i} = \dfrac{(z-2i)(z+2i)}{z-2i} = z+2i $.
    	\begin{enumerate}[label=\alph*.)]
    		\item We must show that for any $ \epsilon > 0 $, we can find $ \delta > 0 $ such that $ |z+2i - L| < \epsilon $ when $ 0 < |z-i| < \delta $. In this case $ L = 3i $. So we have 
    		\[
    			|z+2i-3i| < \epsilon \Rightarrow |z-i| < \epsilon
    		\]
    		So if we take $ \delta = \epsilon $, the definition is fulfilled. Thus $ \lim\limits_{z\to i} f(z) $ exists and $ \lim\limits_{z\to i} f(z) = 3i $.
    		
    		\item No, if we compute this limit as $ z \to 0 $ on the real and on the imaginary axis, we will see that for the real axis, $ \lim_{x\to 2i} f(z) = (4+y)i $. When we compute the limit on the imaginary axis, we see that $ \lim_{y\to 2i} = x+4i $. Thus the limit depends on the direction in which we approach $ 2i $, which implies that $ f(z) $ is not continuous at $ z=2i $.
    	\end{enumerate}
    \end{prob}

	\begin{prob}{2.142}
		\begin{align*}
			\tan(z) = \frac{\sin(z)}{\cos(z)} &= \frac{\sin(x)\cosh(y) + i \cos(x)\sinh(y)}{\cos(x)\cosh(y) - i \sin(x)\sinh(y)} \cdot \frac{\cos(x)\cosh(y) + i \sin(x)\sinh(y)}{\cos(x)\cosh(y) + i \sin(x)\sinh(y)} \\
			& = \frac{\sin(x)\cos(x) + i \sinh(y)\cosh(y)}{\cosh^2(y) - \sin^2(x)} \\
			& = \frac{\sin(2x) + i \sinh(2y)}{2\cosh^2(y) - 2\sinh^2(x) + 1 - 1} \\
			& = \frac{\sin(2x) + i \sinh(2y)}{\cos(2x) + \cosh(2y)} \\
			& \therefore \boxed{ u(x,y) = \frac{\sin(2x)}{\cos(2x) + \cosh(2y)} \, , \, v(x,y) = \frac{\sinh(2y)}{\cos(2x) + \cosh(2y)} }
		\end{align*}
	\end{prob}

	\begin{prob}{3.45} $ $ \vspace{2mm} \\
		$ f(z) = |z|^2 = x^2+y^2 $
		$$ \dfrac{\partial u}{\partial x} = 2x \, , \, \dfrac{\partial u}{\partial y} = 2y \, , \, \dfrac{\partial v}{\partial x} = 0 \, , \, \dfrac{\partial v}{\partial y} = 0 $$
		
		By the Cauchy-Riemann equations, $ \partial u / \partial x \neq \partial v / \partial y $ and $ \partial v / \partial x \neq - \partial u / \partial y $. Thus $ f(z) $ is not differentiable anywhere.
	\end{prob}

	\begin{prob}{3.48} $ $ \vspace{2mm} \\
		$ f(z) = x^2 + iy^3 $
		$$ \dfrac{\partial u}{\partial x} = 2x \, , \, \dfrac{\partial u}{\partial y} = 0 \, , \, \dfrac{\partial v}{\partial x} = 0 \, , \, \dfrac{\partial v}{\partial y} = 3y^2 $$
		
		By the Cauchy-Riemann equations, $ \partial v / \partial x = - \partial u / \partial y $, but $ \partial u / \partial x \neq  \partial v / \partial y $. Thus $ f(z) $ is not analytic.
	\end{prob}

	\begin{prob}{3.50} $  $
		\begin{enumerate}[label=\alph*.)]
			\item $ u(x,y) = 2x(1-y) $ 
			
			$$ \dfrac{\partial u}{\partial x} = 2(1-y), \dfrac{\partial u}{\partial y} = -2x \Rightarrow \dfrac{\partial^2 u}{\partial x^2} = \dfrac{\partial^2 u}{\partial y^2} = 0 $$ 
			
			So $ u(x,y) $ satisfies Laplace's equation and is therefore harmonic.
			
			\item By the first Cauchy-Riemann equation,
			\[
				\frac{\partial u}{\partial x} = \frac{\partial v}{\partial y} \Rightarrow 2(1-y) = \frac{\partial v}{\partial y} \Rightarrow v(x,y) = \int 2(1-y) \,dy = 2y - y^2 + h(x)
			\]
			And by the second Cauchy-Riemann equation,
			\[
				-\frac{\partial u}{\partial y} = \frac{\partial v}{\partial x} \Rightarrow 2x = h'(x) \Rightarrow h(x) = x^2
			\]
			\[
				 \therefore \boxed{v(x,y) = 2y-y^2+x^2}
			\]
			
			\item We know that $ f(z) = u(x,y) + iv(x,y) $. So
				\begin{align*}
						 f(z) &= 2x(1-y) + i(2y - y^2 + x^2) \\
						 &= 2x - 2xy + i2y - iy^2 + ix^2 \\
						 &= ix^2 - iy^2 - 2xy + 2x + i2y \\
						 &= i(x^2-y^2 + i2xy) + 2(x+iy) \\
						 &= \boxed{iz^2 + 2z} 
				\end{align*}
		\end{enumerate}
	\end{prob}




















    
    
\end{document}