\documentclass[11pt]{article}

\usepackage{amsmath,amssymb}
\usepackage[margin=1in]{geometry}
\usepackage{tikz}

\title{STAT 211 Lecture Notes}
\author{Lukas Zamora}
\date{}

\setlength\parindent{0pt}

\begin{document}
	
	\maketitle
	
	\section{Data Collection and Summarization}
	
	\underline{Statistics}: Science of collecting, classifying, and interpreting data.
	
	\subsection*{Collecting Data}
	
	\begin{itemize}
		\item Observational Study
			\begin{itemize}
				\item Observe a group and measure quantities of interest.
				\item Passive data collection.
				\item Describes the group.
			\end{itemize}
		\item Experimental Study
			\begin{itemize}
				\item Deliberately impose treatments on groups in order to observe responses.
				\item Purpose is to study whether the treatments cause a change in the responses.
			\end{itemize}
	\end{itemize}

	\underline{Population}: The entire group of interest.\\
	
	\underline{Sample}: A part of the population selected to draw conclusions about the entire population.\\
	
	\underline{Census}: A sample that attempts to include the entire population.\\
	
	\underline{Parameter}: A concept that describes the population.\\
	
	\underline{Statistic}: A number produced from a sample that estimates a population parameter.\\
	
	$ \bar{x} $ - \textit{sample statistic}\\
	$ \mu $ - \textit{mean statistic}\\
	
	\underline{Experimental Group}: A collection of experimental units subjected to a difference in treatment, imposed by the experimenter.\\
	
	\underline{Control Group}: A collection of experimental units subjected to the same condition as those in an experimental group that no treatment is imposed.\\
	
	
	
	
	
	
	
	
	
	
	
	
	
	
	
	
	
	
	
	
	
	
	
	
	
	
	
	
	
	
	
	
\end{document}